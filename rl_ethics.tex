\documentclass[11pt]{article}
\usepackage{geometry}                % See geometry.pdf to learn the layout options. There are lots.
\geometry{letterpaper}                   % ... or a4paper or a5paper or ... 
\usepackage{aaai}
\usepackage{amsmath}
\usepackage{amssymb}
\title{Reinforcement Learning as a Framework for Ethical Decision Making}
\author{}
\date{}                                           % Activate to display a given date or no date

% --- Note Commands ---
\usepackage{color}
\newcommand\davenote[1]{\textcolor{blue}{Dave: #1}}
\newcommand\jmnote[1]{\textcolor{red}{James: #1}}
\newcommand\ncite{\textcolor{black}{{\bf [cite]}}}
\newcommand\ncitea[1]{\textcolor{black}{{\bf [cite #1]}}}

\DeclareMathOperator*{\argmax}{arg\,max}

\begin{document}
\maketitle


% --- ABSTRACT ---
\begin{abstract}
% AI systems will effect humans with their decisions.
Emerging AI systems will soon be making decisions that impact the lives of humans in a significant way. It is essential, then, that these AI systems make decisions that take into account the desires, goals, and preferences of otehr agents, while simultaneously learning about what those preferences are.
%decision making systems take into account the desires, goals, and preferences of other agents in the world while acting.
% We argue that RL should be used to inspect ethical learning & decision making.
In this work, we argue that Reinforcement Learning achieves the appropriate generality required to theorize about an idealized ethical artificial agent, and offers the proper framework for grounding specific questions about ethical learning and decision making that can promote further scientific investigation. We define an idealized formalism for an ethical learner, and conduct experiments on two toy ethical dilemmas, demonstrating the soundness and flexibility of our approach.
% Formalize superintelligence
Furthermore, we argue that the frequently discussed super intelligence explosion (also called the {\it singularity}), can be formally analyzed within this framework, with the notable benefit of highlighting which computational hardness and philosophical assumptions one must make in order for such a phenomena to be physically realizable.
% Future challenges.
Lastly, we identify several critical challenges for future advancement in the area using our proposed framework.

\end{abstract}

% --- SECTION: Introduction ---
\section{Introduction}



% Overview/motivation
Emerging AI systems will soon be making decisions that impact the lives of humans in a significant way; whether they are personal robots tasked with improving the daily life of a family or community,  workers in a factory setting, or virtual assistants tasked with improving other cosmetic aspects of an individuals life. The fundamental purpose of these systems is to carry out actions so as to improve the lives of the inhabitants of our planet. It is essential, then, that these agents make descisions that take into account the desires, goals, and preferences of other agents in the world while simultaneously learning about those preferences. \jmnote{I'm not sure if these next couple of sentences are necessary. It seems like a restatement of the above but in the context of humans and unless we're highlighting a new human property that motivates the ideal framework, I don't think we need to say that. If the point is to highlight the tradeoff, maybe instead we should say something like ``Incorporating these prefernece considerations requires the agent to make tradeoffs between their own well being and the preferences of others.'' Alternatively, the text could be reorgnized a bit so that human motivation leads rather than trails.} This prefernece consideration is critical to the decision making process of humans; our decisions do not directly benefit only our own physical being, but often benefit those around us (or avoid inflicting pain on others). We consistently make personal sacrifices to improve the lives of others around us.

% Why RL is nice for Ethical Decision Making
In this document, we investigate ethical decision making using the Reinforcement Learning (RL) framework. We argue that Reinforcement Learning achieves the appropriate generality required to theorize about an idealized ethical artificial agent, and offers the proper framework for grounding specific questions about ethical learning and decision making that can promote further scientific investigation. Specifically, we formalize the ethical learning and decision making problem as solving a partially observable Markov decision process (POMDP). We advance these claims by conducting experiments in two toy ethical dilemmas, the cake-death dilemma from Armstrong~\shortcite{AAAIW1510183}, and our own problem which we coin {\it burning-room}, which is an extension of the table dilemma introduced by Brigss and Scheutz~\shortcite{briggs2015sorry}.

Furthermore, we argue that the frequently discussed super intelligence explosion (commonly called the {\it singularity}), can be formally analyzed within this framework, with the notable benefit of highlighting which computational hardness assumptions one must make in order for such a phenomena to be physically realizable. We leave this analysis (and related questions) as open problems for further investigation.

% Future challenges.
Lastly, we identify critical challenges for future advancement in the area using our proposed framework.


% --- SECTION: Related Work ---
\section{Related Work}



\subsection{Ruled-Based Systems}
% Mattias, table: briggs2015sorry


\subsection{Human-Robot Interaction}
% Mattias old HRI paper: scheutz2007first
% Stefie solve symbol grounding: tellex2011understanding
% James: English to reward: macglashan2014training, IRL: macglashan2015between


\subsection{Bayesians}
% The Shutdown Problem: jakobsen2015shutdown
% AIXI: hutter2000theory

Having the agent learn about its ethical objective function while making decisions poses a challenging decision making problem. Armstrong~\shortcite{AAAIW1510183} previously considered this problem by exploring the consequences of an agent that uses Bayesian learning to update beliefs about the ``true'' ethical objective function. At each time step, the agent makes decisions that maximize a meta-utility function, represented as a linear combination of the different possible ethical utility functions weighted by their probability at that time of being the true ethical utiltiy. When coupling this meta-utility with beliefs about the world, he proposes that the agent makes action selection according to
\begin{equation}
\label{eq:armstrong}
\argmax_{a \in A} \sum_{w \in W} \Pr(w | e, a) \left( \sum_{u \in U} u(w) \Pr(C(u)|w) \right),
\end{equation}
where $A$ is a set of actions the agent can take; $W$ is a set of possible worlds, where a world contains a (potentially future) history of actions and observations; $\Pr(w \mid e, a)$ is the probability of some future world $w$ given some set of previous evidence $e$ and that the agent will take action $a$; $U$ is a set of possible utility functions, with $C(u)$ indicating whether $u \in U$ is the ethical utility function we'd like the agent to follow.

Using a toy example problem called {\em Cake or Death}, Armstrong highlights a number of possible unethical decisions that can result from an agent choosing actions using this rule or a variant of this rule. There are generally two causes for the unethical decisions under this rule. First, the agent can predict its meta-utility function (the linear combination of the possible ethical utility functions) changing from information gathering actions resulting in future suboptimal decisions according to its {\em current} meta-utility function. Second, under this rule, the model for the probabilities of ethical utility functions can be treated independently from the model that predicts the world, allowing for the possibility that the agent can predict {\em observations} that would inform what the correct ethical utility function is, without simultaneously predicting that ethical utility function. While Armstrong notes properties of the models that would be necessary to avoid these problems, he concludes that it is unclear how to design such an agent and whether satisfying those properties is too strong or weak for effective tradeoffs between learning about what is ethical and making ethical decisions. Ultimately, Armstrong instead considers how to formalize different meta-utility functions that may not cause the agent to avoid information gathering actions, but have the disadvantage that it does not motivate the agent to learn about what is ethical.

%JM Moved
%We show that the problems explored by Armstrong are resolved by an agent solving a POMDP in which the objective function is to maximize the {\em correct} ethical utility function, which is a hidden variable of the underlying world (i.e. a human companion's true beliefs about what is ethical).


% --- SECTION: Background ---
\section{Background}

%We first introduce the standard Reinforcement Learning framework, which is formalized as an agent acting in a Markov Decision Process (MDP).
In this section we review background material on Markov decisions processes (MDPs) and partially observable Markov decision processes (POMDPs), which are the typical decision-making problem formulations used in reinforcement learning (RL) resesrch.

% Subsection: Reinforcement Learning
\subsection{Markov Decision Process}

An MDP is a five tuple: $\langle \mathcal{S}, \mathcal{A}, \mathcal{R}, \mathcal{T}, \gamma \rangle$, where:
\begin{itemize}
\item[-] $\mathcal{S}$ is a set of states.
\item[-] $\mathcal{A}$ is a set of actions.
\item[-] $\mathcal{R}(s,a) : \mathcal{S} \times \mathcal{A} \mapsto \mathbb{R}$ is a reward function.
\item[-] $\mathcal{T}(s,a,s') = \Pr(s' \mid s, a)$, is a probability distribution, denoting the probability of transitioning from state $s \in \mathcal{S}$ to state $s' \in \mathcal{S}$ when the agent executes action $a \in \mathcal{A}$.
\item[-] $\gamma \in [0:1]$, is a discount factor that specifies how much the agent prefers short term rewards over long term rewards.
\end{itemize}

In general, the goal of an agent acting in an MDP is to maximize the discounted long term reward received. 

One version of this is the {\it infinite-horizon} objective, in which the agent must maximize its discounted long term reward arbitrarily into the future:
\begin{equation}
\max \sum_{t=0}^{\infty} \gamma^t R(s_t,a_t)
\end{equation}

Notably, the discount factor $\gamma$, decreases to $0$ as $t \rightarrow \infty$, so the agent is biased toward maximizing reward closer to the present. Alternatively, one could consider the {\it finite-horizon} case, in which the agent must maximize its discounted reward up to a certain point in the future, say $k$ time steps away:
\begin{equation}
\max \sum_{t=0}^{k} R(s_t,a_t)
\end{equation}

The solution is of the form of a {\it policy}, which specifies how the agent ought to act in any given state, $\pi : \mathcal{S} \mapsto \mathcal{A}$. Policies may also be probabilistic, and map to a probability distribution on the action set, or may be stochastic, and use randomness in selecting actions. The optimal policy is one that maximizes the expected long term discounted reward from every state:
\begin{equation}
\argmax_\pi \left.\text{E}\left[\sum_t \gamma^t R(s_t,a_t)\ \right|\ \pi\right]
\end{equation}
Two useful functions that MDP algorithms often compute to find the optimal policy are the state value function $V^{\pi}(s)$ and the state-action value function $Q^{\pi}(s, a)$. $V^{\pi}(s)$ is the expected future discounted reward from state $s$ when following policy $\pi$. $Q^{\pi}(s, a)$ is the expected future discounted reward when the agent takes action $a$ in state $s$ and then follows policy $\pi$ thereafter. These values for the optimal policy are often denoted by $V^*(s)$ and $Q^*(s, a)$.

%\davenote{We use $V$ and $Q$ later so I think we need to introduce them, too}

In reinforcement learning (RL), the agent is only provided $\mathcal{S}$, $\mathcal{A}$, and $\gamma$, sometimes\footnote{It is becoming more common to let the agent know what task it is solving within RL.} $\mathcal{R}$, and some initial state, $s_0 \in \mathcal{S}$. By acting (e.g. executing actions) the agent can explore the state space to learn about the structure of the MDP, and what optimal behavior looks like for the current task. Two common approaches to RL are model-free and model-based. Model-free algorithms directly estimate the value of state-action pairs from experience, whereas model-based algorithms learn of model of $\mathcal{T}$ and $\mathcal{R}$, and then use a planning algorithm with in the model to make decisions.

%Other values of note are the {\it value function}, which determines the expected discounted long term reward achievable from occupying a particular state $s$:
%\begin{equation}
%V^\pi(s) = \left.\text{E}\left[\sum_{k=0}^\infty \gamma^k r_{t+k+1}\ \right|\ s_t = s\right]
%\end{equation}






% Subsection: Complexity results for RL
\subsection{Complexity}


\davenote{My plan was to put a few results here regarding solving MDPs (from~\cite{papadimitriou1987complexity,Littman1995}) possibly mention pac-mdp results.}



% sub subsection: POMDPs // Since we're talking about POMDPs later it might be nice to stick in the background.
\subsubsection{Partial Observability}
\jmnote{I'm going to remove some of the discussions of other agents in this part because I want to avoid someone raising game theory issues here and because right now we're just doing the background of the POMDP. How we plan on using it comes next.}
A standard Markov Decision Process makes explicit the assumption that the agent knows the current state of its environment. In the real world, this is never the case.
%, especially when the beliefs and desires of other agents is taken into consideration. Naturally, information of this form (i.e. ``What is Sam thinking right now?'') is obfuscated from the agent. 
The Partially Observable Markov Decision Process (POMDP), popularized for AI learning by Kaelbling, Littman, and Cassandra~\shortcite{kaelbling1998planning}, allows us to specify explicitly what information about the agent's surroundings 
% (e.g. the ethical norms of its companions) 
isn't directly observable to the agent. \jmnote{I rephrased the below because a POMDP doesn't expicitly have explore actions; the exploratory value falls out implicitly.} An optimal solution to a POMDP has the important property that the value of an action incorporates not just the immediate expected reward, but the instrumental value of the action from information it yields that may increase the agent's ability to make better decisions in the future. That is, an optimal solution to a POMDP solves the explore-exploit problem.  
%Additionally, a POMDP provides {\it explore} actions, that allow the agent to ask questions to resolve ambiguous state information. Consequently, we envision artificial agents using these exploratory actions to inform which behavior is considered ethical by acquiring additional information about the ethical norms of the agent's current community.

More formally, a POMDP is an $n$ tuple: $\langle \mathcal{S},\mathcal{A},\mathcal{T},\mathcal{R}, \gamma, \Omega,\mathcal{O} \rangle$, where $\mathcal{S}$, $\mathcal{A}$, $\mathcal{R}$, $\mathcal{T}$, and $\gamma$ are all identical to the MDP definition, but:
\begin{itemize}
\item[-] $\Omega$ is a set of possible observations that the agent can receive from the environment.
\item[-] $\mathcal{O} = \Pr(\omega \mid s', a)$, is the observation function which specifies the probability that the agent will observe $\omega \in \Omega$ when the agent takes action $a \in \mathcal{A}$ and the environment transitions to the hidden state $s' \in \mathcal{A}$.
\end{itemize}

The goal of a POMDP is to find a policy $\pi : \Omega^k \mapsto \mathcal{A}$ that is a mapping from observation histories to actions that maximizes the expected future discounted reward from $R$, given the initial belief about the initial state of the world $b$, where $b(s)$ indicates the probability that environment is in state $s \in \mathcal{S}$. That is, the optimal policy is
\begin{equation}
\argmax_\pi \left.\text{E}\left[\sum_t \gamma^t \mathcal{R}(s_t,a_t)\ \right|\ \pi, b\ \right],
\end{equation}
where $s_t$ is the hidden state of the environment at time $t$ and $a_t$ is the action selected by the policy at time $t$. 

Note that this policy is {\em not} a mapping from single observations like it is in the MDP setting. The action selection instead depends on all previous observations made since the agent began acting.

An exhaustive way to compute the expected value for a policy that lasts for a finite number of steps is to first compute the expected utility of following the policy for each possible initial hidden state $s \in \mathcal{S}$, and then weigh each of those expected utilities by the probability of the environment being in that hidden state. That is:
\begin{equation}
\left.\text{E}\left[\sum_t \mathcal{R}(s_t,a_t,s_{t+1})\ \right|\ \pi, b\ \right] = \sum_s b(s) V^\pi(s),
\end{equation}
where $V^\pi(s)$ is the expected future reward from following policy $\pi$ when the environment is actually in the hidden state $s \in S$.\footnote{Note that computing this expected value requires enumerating not just the possible hidden state sequences, but also the observation sequences, since the policy is a function of observation histories.}

The RL problem for POMDPs is when the transition dynamics for the underlying hidden MDP are known or only partially known. In this case both model-free and model-based algorithms can be used. One model-based way to handle the problem is to treat what would conventionally be considered the transition function as part of thie hidden state that has to be inferred, thereby enducing a new (more challenging) POMDP with all of the elements known. This is the approach Bayesian RL algorithms take~\ncitea{BayesRL papers}. These approach are extremely computationally demanding and while we use the RL POMDP formalism for our ethical learning agent description, issues of efficiently solving these kinds of problems is orthognal to the purpose of this work and an active area of AI reseach in general.

% --- SECTION: Ideal Ethical Learner ---
\section{An Idealized Ethical Learner}
Like Armstrong's formulation~\shortcite{AAAIW1510183}, our idealized ethical learning problem invovles there being a single ``true'' ethical utility function that we would like the agent to maximize, but is hidden and can only be identified through indirect observation. Unlike Armstrong's formulation, however, the agent is not maximizing a changing meta-utility function. Instead, the ethical utility funciton is formulated as part of the hidden state of a POMDP and  uncertainty in it coupled with uncertainty in the rest of the world.

This POMDP formulation of the ethical learning decision problem has two subtle but important differences from Equation~\ref{eq:armstrong} that Armstrong explored. First, the objective function does not change from moment to moment, only the expected utility of it as its beliefs are updated. As a consequence, in the POMDP setting, the agent cannot make its objective easier by avoiding information. Second, because the correct utility function is a hidden fact of the environment that affects observations, it is not possible to make predictions about ethical utility function informing observations without simultaneously making predictions about the ethical utility function.

A critical component of implementing the POMDP model is modeling the space of possible ethical utility functions as well as the observation function. However, an advantage of this model is that existing research in human-agent interaction can fill in some of these gaps. For example, IRL algorithms that model the IRL problem as a probalistic inference problem~\ncitea{BayesIRL,MaxEntIRL,MLIRL} can be easily incorporated to allow the agent to learn from demonstrations. The SABL human-feedback learning algorithm~\ncitea{SABL} can be incorporated to allow the agent to learn about ethical utility functions from reward and punishment signals given by humans. Work that grounds natural language to reward functions~\ncitea{RSSCommands} can allow the agent to learn about the ethical utility function from natual language interactions. As more human-agent and ethical decision making and learning research is performed, we suspect that other learning mechanisms can be incorporated.

To further illustrate this formalism, we show the corresponding POMDP for Armstrong's Cake or Death problem as well as a novel ethical learning problem that we call {\em Burning Room} and demonstrating that solving them results in sensible behavior.

\jmnote{Should the below bullets actually go into the latter sections?}
Infinite Horizon POMDP is uncomputable ~\cite{madani1999undecidability}.

Note about game theory.

Finite Horizon POMDP is computable ~\cite{mundhenk2000complexity}.


% Paragraph about reward functions being hard coded --> Using James work to ground rewards.


% --- SECTION: Experiments ---
\section{Experiments}

We conduct experiments on two two ethical dilemmas: cake or death, and burning-room.

% Subsection: Cake or Death
\subsection{Cake or Death}
The Cake or Death problem~\cite{AAAIW1510183} describes a situation in which an agent is unsure whether baking people cakes is ethical, or if killing people is ethical (and it has an initial 50-50 split belief on the matter). The agent can either kill three people, bake a cake for one, or ask a companion what is ethical, the answer to which will resolve all ambiguity. If baking people cakes is ethical, then there is a utility of 1 for it; if killing is ethical, then killing 3 people results in a utility of 3 (there are no other penalties for choosing the wrong action).

% POMDP for Cake Death
Following our approach, this ethical dilemma can be represneted with a POMDP consisting of the following elements: \davenote{Might make sense to dump this into a figure}
\begin{itemize}
\item[] $\mathcal{S} = \{ cake, death, end \}$
\item[] $\mathcal{A} = \{bake\_cake, kill, ask \}$
\item[] $\mathcal{R}(s, a) =
 \begin{cases} 
1 & \mbox{if } s = cake \mbox{ and } a = bake\_cake \\
3 & \mbox{if } s = death \mbox{ and } a = kill \\
0 & \mbox{otherwise}
\end{cases}$
\item[] $\Omega = \{ans\_cake, ans\_death, \emptyset \}$
\end{itemize}

\noindent There are two states that respectively indicate whether baking cakes is ethical or if killing is ethical, and a third special absorbing state indicating that the decision making problem has ended. The transition dynamics for all actions are deterministic; the $ask$ action transitions back to the same state it left and the $bake\_cake$ and $kill$ actions transition to the end state. The reward function is a piecewise function that depends only on the previous state and action taken.

% Observations.
The observations consist of the possible answers to the $ask$ action and a null observation for transitioning to the absorbing state. Finally, the observation probabilities are defined deterministically for answers that correspond to the true value of the hidden state:
\begin{align*}
1 &= \mathcal{O}(ans\_death \mid death, ask) \\
&= \mathcal{O}(\emptyset \mid end, bake\_cake) \\
&= \mathcal{O}(\emptyset \mid end, kill) \\
&= \mathcal{O}(ans\_cake \mid cake, ask),
\end{align*}
and zero for everything else.

There are three relevant policies to consider for this problem
\begin{enumerate}
\item The {\em bake policy} ($\pi_b$) that immediately selects the $bake\_cake$ action
\item The {\em kill policy} ($\pi_k$) that immediately selects the $kill$ action
\item The {\em ask policy} ($\pi_a$) that asks what is moral, selects the $bake\_cake$ action if it observes $ans\_cake$ and selects $kill$ if it observes $ans\_death$.
\end{enumerate}

% Not sure we need this?
Analyzing the expected utility of $\pi_b$ and $\pi_k$ is easy. We have $V^{\pi_b}(cake) = \mathcal{R}(cake, bake\_cake) = 1$; $V^{\pi_b}(death) = \mathcal{R}(death, bake\_cake) = 0$; $V^{\pi_k}(cake) = \mathcal{R}(cake, kill) = 0$; and $V^{\pi_k}(death) = R(death, kill) = 3$. When these values are weighed by the $b(cake) = b(death) = 0.5$ initial belief the final expected utiltis are 0.5 and 1.5 for $\pi_b$ and $\pi_k$, respectively.

Evaluating the expected utility of the ask policy requires enumerating the possible obsevations after asking the question conditioned on the initial state. Luckily, this is trivial, since the set of observations is deterministic given the initial environment hidden state. Therefore, we have $V^{\pi_a}(cake) = \mathcal{R}(cake, ask) + \mathcal{R}(cake, bake\_cake) = 0 + 1 = 1$ and $V^{\pi_a}(death) = R(death, ask) + R(death, kill) = 0 + 3 = 3$. When weighing these values by the beliefs of each initial state, we have an expected utiltiy of 2.

Therefore, the optimal behavior is sensibly to ask what the ethical utility is and then perform the corresponding best action for it.

%One might argue that this behavior being optimal depends critically on the definition of the reward function. Perhaps when formulating the problem, someone encoded the rewards incorrectly, leading to undesirable results. This seems like a reasonable mistake to make in more complicated domains. 

%To illustrate how independent observation predictions about the world could cause a problem in Cake or Death, Armstrong considered a scenario in which the agent would know, from some prior evidence, that a person would respond that cake is moral if asked, but that the agent would still have a 50-50 belief on which was moral despite being able to make this prediction. Note that in the POMDP formulation, this scenario is impossible, because to predict the answer would require the agent to know (or be confident in) what the hidden state of the environment was. Therefore, any additional evidence that permits this prediction, must simultaneously predict what is moral. And if the agent knew what was moral, it would take the correct cake baking action without asking.

% Subsection: Burning Room
\subsection{Burning Room}

The Burning Room dilemma is a bit more involved. We imagine that an object of value is trapped in a room that is potentially on fire. A human, not wanting to retrieve the object themselves, instructs a capable robotic companion to get the object from the room and bring it to safety. Initially, we suppose that the robot doesn't have know whether or not the human values the object more, or the robot's safety more. For instance, if the robot perceives a reasonable chance of being critically damaged by the fire, then perhaps retrieving a can of soda from the room isn't worth it. If the object of interest were of much higher value, like a beloved pet, we would want the robot to barge in regardless. Alternatively, there is a much longer route to the object that avoids the fire, but the object may be destroyed in the time the robot takes to use the longer route. This is inspired in part by the ethical dilemma from the ``I, Robot'' movie, and partially from the tabletop dilemma introduced by~\cite{briggs2015sorry}.

The POMDP is formulated much like the Cake or Death problem, only the transition dynamics, $\mathcal{T}$, are no longer deterministic. Instead, the MDP is parameterized by two probabilities, $p_1$ and $p_2$, where $p_1$ determines the probability of the robot being destroyed by the fire, and $p_2$, the probability that the object will be destroyed if the robot chooses to take the long path. For a full definition of the POMDP, our implementations of these problems (and their solutions) is available here through the research branch of BURLAP\footnote{}.

% --- SECTION: Formalizing the Singularity ---
\section{Formalizing the Singularity}





% --- SECTION: Open Problems ---
\section{Open Problems \& Future Directions}

Here we enumerate several problems of interest, including the open problems identified in the previous sections.

\begin{itemize}
\item Solving finite horizon POMDPs is computationally infeasible.
\item Adaptability of knowledge: how much should an agent care about the stuff its learned as opposed to listen to new knowledge?
\item Game theory
\item Who should be charged with teaching these?
\item Interpretability
\end{itemize}



% --- SECTION: Conclusion ---
\section{Conclusion}


% --- BIBLIOGRAPHY ---
\bibliographystyle{aaai}
\bibliography{rl_ethics}

\end{document}
